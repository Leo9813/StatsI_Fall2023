\documentclass[12pt,letterpaper]{article}
\usepackage{graphicx,textcomp}
\usepackage{natbib}
\usepackage{setspace}
\usepackage{fullpage}
\usepackage{color}
\usepackage[reqno]{amsmath}
\usepackage{amsthm}
\usepackage{fancyvrb}
\usepackage{amssymb,enumerate}
\usepackage[all]{xy}
\usepackage{endnotes}
\usepackage{lscape}
\newtheorem{com}{Comment}
\usepackage{float}
\usepackage{hyperref}
\newtheorem{lem} {Lemma}
\newtheorem{prop}{Proposition}
\newtheorem{thm}{Theorem}
\newtheorem{defn}{Definition}
\newtheorem{cor}{Corollary}
\newtheorem{obs}{Observation}
\usepackage[compact]{titlesec}
\usepackage{dcolumn}
\usepackage{tikz}
\usetikzlibrary{arrows}
\usepackage{multirow}
\usepackage{xcolor}
\newcolumntype{.}{D{.}{.}{-1}}
\newcolumntype{d}[1]{D{.}{.}{#1}}
\definecolor{light-gray}{gray}{0.65}
\usepackage{url}
\usepackage{listings}
\usepackage{color}

\definecolor{codegreen}{rgb}{0,0.6,0}
\definecolor{codegray}{rgb}{0.5,0.5,0.5}
\definecolor{codepurple}{rgb}{0.58,0,0.82}
\definecolor{backcolour}{rgb}{0.95,0.95,0.92}

\lstdefinestyle{mystyle}{
	backgroundcolor=\color{backcolour},   
	commentstyle=\color{codegreen},
	keywordstyle=\color{magenta},
	numberstyle=\tiny\color{codegray},
	stringstyle=\color{codepurple},
	basicstyle=\footnotesize,
	breakatwhitespace=false,         
	breaklines=true,                 
	captionpos=b,                    
	keepspaces=true,                 
	numbers=left,                    
	numbersep=5pt,                  
	showspaces=false,                
	showstringspaces=false,
	showtabs=false,                  
	tabsize=2
}
\lstset{style=mystyle}
\newcommand{\Sref}[1]{Section~\ref{#1}}
\newtheorem{hyp}{Hypothesis}

\title{Problem Set 3}
\date{Due: November 19, 2022}
\author{Applied Stats/Quant Methods 1}


\begin{document}
	\maketitle
	\section*{Instructions}
	\begin{itemize}
		\item Please show your work! You may lose points by simply writing in the answer. If the problem requires you to execute commands in \texttt{R}, please include the code you used to get your answers. Please also include the \texttt{.R} file that contains your code. If you are not sure if work needs to be shown for a particular problem, please ask.
	\item Your homework should be submitted electronically on GitHub.
	\item This problem set is due before 23:59 on Sunday November 19, 2023. No late assignments will be accepted.

	\end{itemize}

		\vspace{.25cm}
	
\noindent In this problem set, you will run several regressions and create an add variable plot (see the lecture slides) in \texttt{R} using the \texttt{incumbents\_subset.csv} dataset. Include all of your code.

	\vspace{.5cm}
\section*{Question 1}
\vspace{.25cm}
\noindent We are interested in knowing how the difference in campaign spending between incumbent and challenger affects the incumbent's vote share. 
	\begin{enumerate}
		\item Run a regression where the outcome variable is \texttt{voteshare} and the explanatory variable is \texttt{difflog}.\\
		
		\vspace{.15cm}
        \lstinputlisting[language=R, firstline=39, lastline=41]{PS3_Jie_Yu.R}

		\vspace{.15cm}
		
		\begin{footnotesize}
			
			\begin{verbatim}
			Call:
			lm(formula = voteshare ~ difflog, data = inc.sub)
			
			Residuals:
			Min       1Q   Median       3Q      Max 
			-0.26832 -0.05345 -0.00377  0.04780  0.32749 
			
			Coefficients:
			Estimate Std. Error t value Pr(>|t|)    
			(Intercept) 0.579031   0.002251  257.19   <2e-16 ***
			difflog     0.041666   0.000968   43.04   <2e-16 ***
			---
			Signif. codes:  0 ‘***’ 0.001 ‘**’ 0.01 ‘*’ 0.05 ‘.’ 0.1 ‘ ’ 1
			
			Residual standard error: 0.07867 on 3191 degrees of freedom
			Multiple R-squared:  0.3673,	Adjusted R-squared:  0.3671 
			F-statistic:  1853 on 1 and 3191 DF,  p-value: < 2.2e-16
			
			According to the results, the F statistic is 1853, 
			the degrees of freedom are 1 and 3191,
			and the p-value is less than 2.2e-16,
			indicating that the overall model is highly significant 
			and we have high confidence to reject the null hypothesis.
		
			\end{verbatim}
		\end{footnotesize}
		\noindent 
		\item Make a scatterplot of the two variables and add the regression line. 		
		\vspace{.15cm}
		\lstinputlisting[language=R, firstline=44, lastline=52]{PS3_Jie_Yu.R} 
		\vspace{.15cm}
		\begin{figure}[h!]\centering
			
			\caption{\footnotesize Scatterplot of relationship between \texttt{voteshare}  and  \texttt{difflog}.}
			\label{fig:Rplot_1}
			\includegraphics[width=.85\textwidth]{Rplot_1.pdf}
		\end{figure}
	\begin{footnotesize}
		\begin{verbatim}
			According to the results, the linear correlation is positive, 
			indicating that as campaign spending increases, 
			the incumbent's vote share tends to increase.
			Moreover, the dotted plots are concentrated on both sides of the linear plot, 
			indicating that there is a strong linear relationship.
			\end{verbatim}
	\end{footnotesize}
		\vspace{80.15cm}
		\item Save the residuals of the model in a separate object.
		
		\vspace{.15cm}
		\lstinputlisting[language=R, firstline=56, lastline=59]{PS3_Jie_Yu.R} 
		\vspace{.15cm}
	\begin{footnotesize}
		\begin{verbatim} 
		Save the residuals in a separate object:
		Named num [1:3193] -0.000423 -0.031684 -0.004551 0.038669 0.035529 ...
		- attr(*, "names")= chr [1:3193] "1" "2" "3" "4" ...
		Length  Class   Mode 
		0   NULL   NULL 
		\end{verbatim}
	\end{footnotesize}
		\item Write the prediction equation.
		
		\vspace{.15cm}
		\lstinputlisting[language=R, firstline=63, lastline=68]{PS3_Jie_Yu.R} 
		\vspace{.15cm}
		\vspace{.15cm}
	\begin{footnotesize}
		\begin{verbatim}
	  Final Model: votehare= 0.579 + 0.0417 * difflog
	  
	  According results,
	  For each one-unit increase in difflog, 
	  we expect voteshare to increase by approximately 0.0417, 
	  assuming all other factors remain constant. 
	  The intercept of 0.579 represents the estimated voteshare when difflog is zero.
			\end{verbatim}
	\end{footnotesize}
	\end{enumerate}
	
\vspace{.5cm}

\section*{Question 2}
\noindent We are interested in knowing how the difference between incumbent and challenger's spending and the vote share of the presidential candidate of the incumbent's party are related.	\vspace{.25cm}
	\begin{enumerate}
		\item Run a regression where the outcome variable is \texttt{presvote} and the explanatory variable is \texttt{difflog}.	
		
		\vspace{.15cm}
		\lstinputlisting[language=R, firstline=75, lastline=77]{PS3_Jie_Yu.R} 
		\vspace{.15cm}
	\begin{footnotesize}
		\begin{verbatim} 
	Call:
	lm(formula = presvote ~ difflog, data = inc.sub)
	
	Residuals:
	Min       1Q   Median       3Q      Max 
	-0.32196 -0.07407 -0.00102  0.07151  0.42743 
	
	Coefficients:
	Estimate Std. Error t value Pr(>|t|)    
	(Intercept) 0.507583   0.003161  160.60   <2e-16 ***
	difflog     0.023837   0.001359   17.54   <2e-16 ***
	---
	Signif. codes:  0 ‘***’ 0.001 ‘**’ 0.01 ‘*’ 0.05 ‘.’ 0.1 ‘ ’ 1
	
	Residual standard error: 0.1104 on 3191 degrees of freedom
	Multiple R-squared:  0.08795,	Adjusted R-squared:  0.08767 
	F-statistic: 307.7 on 1 and 3191 DF,  p-value: < 2.2e-16
	
	According to the results, the F statistic is 307.7, 
	the degrees of freedom are 1 and 3191, 
	and the p-value is less than 2.2e-16, 
	indicating that the overall model is highly significant 
	and has high confidence to reject the null hypothesis.
	
			\end{verbatim}
	\end{footnotesize}
		\vspace{.15cm}
		\item Make a scatterplot of the two variables and add the regression line. 	
		\vspace{.15cm}
		\lstinputlisting[language=R, firstline=80, lastline=88]{PS3_Jie_Yu.R} 
		\vspace{.15cm}
		\begin{figure}[h!]\centering
			
			\caption{\footnotesize Scatterplot of relationship between \texttt{presvote}and \texttt{difflog}.}
			\label{fig:Rplot_2}
			\includegraphics[width=.85\textwidth]{Rplot_2.pdf}
		\end{figure}
		\begin{footnotesize}
			\begin{verbatim} 
			According to the results, the linear correlation is positive, 
			indicating that campaign spending increases, 
			the presvote tends to increase.
			Moreover, the dotted plots are concentrated on both sides of the linear plot, 
			indicating that there is a strong linear relationship.
		\end{verbatim}
	\end{footnotesize}
		
		\item Save the residuals of the model in a separate object.	
		\vspace{.15cm}
		\lstinputlisting[language=R, firstline=91, lastline=94]{PS3_Jie_Yu.R} 
	 \begin{footnotesize}
		 \begin{verbatim} 
		 	Named num [1:3193] 0.00561 0.03758 -0.05313 -0.05299 -0.04584 ...
		 	- attr(*, "names")= chr [1:3193] "1" "2" "3" "4" ...
		 	Length  Class   Mode 
		 	0   NULL   NULL
	          
     	 \end{verbatim}
      \end{footnotesize}
		\item Write the prediction equation.
		
		\vspace{.15cm}
		\lstinputlisting[language=R, firstline=97, lastline=102]{PS3_Jie_Yu.R} 
		\vspace{.15cm}
		\begin{footnotesize}
			\begin{verbatim}  
		Final Model: presvote= 0.5076 + 0.0238 * difflog
	    
		Based on the results, for each unit increase in the difflog variable,
		we expect the pre-voting to increase by approximately 0.0238.
		The intercept of 0.5076 represents the estimated baseline value of presvote 
		when there is no change in the difflog variable.
        	\end{verbatim}
        \end{footnotesize}
	\end{enumerate}
	
\vspace{.5cm}
\section*{Question 3}

\noindent We are interested in knowing how the vote share of the presidential candidate of the incumbent's party is associated with the incumbent's electoral success.
	\vspace{.25cm}
	\begin{enumerate}
		\item Run a regression where the outcome variable is \texttt{voteshare} and the explanatory variable is \texttt{presvote}.
			
			\vspace{.15cm}
			\lstinputlisting[language=R, firstline=106, lastline=108]{PS3_Jie_Yu.R}
			\begin{footnotesize}
				\begin{verbatim}
					Call:
					lm(formula = voteshare ~ presvote, data = inc.sub)
					
					Residuals:
					Min       1Q   Median       3Q      Max 
					-0.27330 -0.05888  0.00394  0.06148  0.41365 
					
					Coefficients:
					Estimate Std. Error t value Pr(>|t|)    
					(Intercept) 0.441330   0.007599   58.08   <2e-16 ***
					presvote    0.388018   0.013493   28.76   <2e-16 ***
					---
					Signif. codes:  0 ‘***’ 0.001 ‘**’ 0.01 ‘*’ 0.05 ‘.’ 0.1 ‘ ’ 1
					
					Residual standard error: 0.08815 on 3191 degrees of freedom
					Multiple R-squared:  0.2058,	Adjusted R-squared:  0.2056 
					F-statistic:   827 on 1 and 3191 DF,  p-value: < 2.2e-16
			\end{verbatim}
		\end{footnotesize}
		
		\item Make a scatterplot of the two variables and add the regression line. 
		
		\vspace{.15cm}
		\lstinputlisting[language=R, firstline=111, lastline=119]{PS3_Jie_Yu.R}
		\vspace{.15cm}
		\begin{figure}[h!]\centering
			
			\caption{\footnotesize Scatterplot of relationship between \texttt{voteshare} and \texttt{presvote}.}
			\label{fig:Rplot_3}
			\includegraphics[width=.85\textwidth]{Rplot_3.pdf}
		\end{figure}
		\begin{footnotesize}
			\begin{verbatim}
			According to the results, the linear correlation is positive, 
			indicating that campaign spending increases, 
			the presvote tends to increase.
			Moreover, the dotted plots are concentrated on both sides of the linear plot, 
			indicating that there is a strong linear relationship.	
				
			\end{verbatim}
	\end{footnotesize}		
		
		\item Write the prediction equation.
		\vspace{.15cm}
		\lstinputlisting[language=R, firstline=122, lastline=127]{PS3_Jie_Yu.R} 
		\begin{footnotesize}
			\begin{verbatim}
				Final Model: voteshare = 0.4413 + 0.388 * presvote
				
				For each unit increase in the prevote variable, 
				we expect the vote share to increase by approximately 0.388. 
				The intercept 0.4413 represents the estimated baseline value of vote share
				without the influence of pre-voting variables.
				
		     \end{verbatim}
	    \end{footnotesize}	
	
		
	\end{enumerate}
	

\vspace{.5cm}
\section*{Question 4}
\noindent The residuals from part (a) tell us how much of the variation in \texttt{voteshare} is $not$ explained by the difference in spending between incumbent and challenger. The residuals in part (b) tell us how much of the variation in \texttt{presvote} is $not$ explained by the difference in spending between incumbent and challenger in the district.
	\begin{enumerate}
		\item Run a regression where the outcome variable is the residuals from Question 1 and the explanatory variable is the residuals from Question 2.	
	
		\vspace{.15cm}
		\lstinputlisting[language=R, firstline=132, lastline=134]{PS3_Jie_Yu.R} 
		\vspace{.15cm}
		\begin{footnotesize}
			\begin{verbatim}
			Call:
			lm(formula = res_1 ~ res_2, data = inc.sub)
			
			Residuals:
			Min       1Q   Median       3Q      Max 
			-0.25928 -0.04737 -0.00121  0.04618  0.33126 
			
			Coefficients:
			Estimate Std. Error t value Pr(>|t|)    
			(Intercept) -5.934e-18  1.299e-03    0.00        1    
			res_2        2.569e-01  1.176e-02   21.84   <2e-16 ***
			---
			Signif. codes:  0 ‘***’ 0.001 ‘**’ 0.01 ‘*’ 0.05 ‘.’ 0.1 ‘ ’ 1
			
			Residual standard error: 0.07338 on 3191 degrees of freedom
			Multiple R-squared:   0.13,	Adjusted R-squared:  0.1298 
			F-statistic:   477 on 1 and 3191 DF,  p-value: < 2.2e-16
			
			According to the results, the F statistic is 477, 
			the degrees of freedom are 1 and 3191,
			and the p value is less than 2.2e-16, 
			indicating a high degree of confidence in rejecting the null hypothesis.
			In summary, the model suggests that there is a 
			statistically significant relationship between res_1 and res_2. 
			The F-statistic and its associated p-value support 
			the overall significance of the model. 
			The coefficient for res_2 is also significant, 
			suggesting that it has a meaningful impact on the dependent variable 
				\end{verbatim}
			\end{footnotesize}	
		\item Make a scatterplot of the two residuals and add the regression line. 	
		\vspace{.15cm}
		\lstinputlisting[language=R, firstline=137, lastline=145]{PS3_Jie_Yu.R} 
		\vspace{.15cm}
		\begin{figure}[h!]\centering
			
			\caption{\footnotesize Scatterplot of relationship between \texttt{residuals\textunderscore1} and \texttt{residuals\textunderscore2}.}
			\label{fig:Rplot_4}
			\includegraphics[width=.85\textwidth]{Rplot_4.pdf}
			
		\end{figure}
		\newpage
		\item Write the prediction equation.
		
		\vspace{.15cm}
		\lstinputlisting[language=R, firstline=149, lastline=154]{PS3_Jie_Yu.R} 
		\vspace{.15cm}
		\begin{footnotesize}
			\begin{verbatim}
				Final Model: res_1= 0 + 0.2569 * res_2
			\end{verbatim}
		\end{footnotesize}	
		
	\end{enumerate}
	
\vspace{.5cm}

\section*{Question 5}
\noindent What if the incumbent's vote share is affected by both the president's popularity and the difference in spending between incumbent and challenger? 
	\begin{enumerate}
		\item Run a regression where the outcome variable is the incumbent's \texttt{voteshare} and the explanatory variables are \texttt{difflog} and \texttt{presvote}.
		
		\vspace{.15cm}
		\lstinputlisting[language=R, firstline=158, lastline=160]{PS3_Jie_Yu.R}
		\vspace{.15cm}
			\begin{footnotesize}
			\begin{verbatim}
				Call:
				lm(formula = voteshare ~ difflog + presvote, data = inc.sub)
				
				Residuals:
				Min       1Q   Median       3Q      Max 
				-0.25928 -0.04737 -0.00121  0.04618  0.33126 
				
				Coefficients:
				Estimate Std. Error t value Pr(>|t|)    
				(Intercept) 0.4486442  0.0063297   70.88   <2e-16 ***
				difflog     0.0355431  0.0009455   37.59   <2e-16 ***
				presvote    0.2568770  0.0117637   21.84   <2e-16 ***
				---
				Signif. codes:  0 ‘***’ 0.001 ‘**’ 0.01 ‘*’ 0.05 ‘.’ 0.1 ‘ ’ 1
				
				Residual standard error: 0.07339 on 3190 degrees of freedom
				Multiple R-squared:  0.4496,	Adjusted R-squared:  0.4493 
				F-statistic:  1303 on 2 and 3190 DF,  p-value: < 2.2e-16
				
				According to the results, the F statistic is 1303, 
				the degrees of freedom are 2 and 3190, 
				and the p-value is less than 2.2e-16, 
				indicating that there is greater credibility to reject the null hypothesis.
				In summary, the model suggests a significant relationship 
				between voteshare, difflog, and presvote. 
				The R-squared values indicate that the model explains 
				a substantial proportion of the variance in voteshare.
			\end{verbatim}
		\end{footnotesize}	
		\item Write the prediction equation.	
		\vspace{.15cm}
		\lstinputlisting[language=R, firstline=163, lastline=170]{PS3_Jie_Yu.R}
		\vspace{.15cm}
		\begin{footnotesize}
			\begin{verbatim}
				Final Model: votehare= 0.4486 + 0.0355 * difflog + 0.2569 * presvote
			\end{verbatim}
		\end{footnotesize}	
		
		\item What is it in this output that is identical to the output in Question 4? Why do you think this is the case?
		\begin{footnotesize}
			\begin{verbatim}
				In both instances, the coefficient for the variable of interest
				 (presvote in Question 4 and residuals2 in the recent regression) 
				 coincidentally appears to be approximately 0.2569. 
				 The associated t-values are both 21.84, 
				 and both exhibit highly significant p-values (<2e-16).
				 This similarity might be a statistical coincidence or could suggest 
				 that the variable residuals2 has a comparable impact 
				 on the dependent variable (voteshare in Question 4 and residuals1 in the recent regression).
				 While investigating further, it is essential to consider
				 whether there is a theoretical or logical explanation 
				 for this observed similarity. 
				 Additionally, careful scrutiny of the model setup 
				 and data appropriateness is crucial. If the relationship 
				 between residuals2 and the dependent variable 
				 is unexpected from a theoretical standpoint, 
				 it may warrant a closer examination to ensure 
				 the robustness of the analysis and rule out potential data issues.
			\end{verbatim}
	     \end{footnotesize}
		\vspace{.15cm}
		\noindent
		\vspace{.15cm}
		
	\end{enumerate}




\end{document}

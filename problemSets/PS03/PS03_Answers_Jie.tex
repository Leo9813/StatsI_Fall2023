\documentclass[12pt,letterpaper]{article}
\usepackage{graphicx,textcomp}
\usepackage{natbib}
\usepackage{setspace}
\usepackage{fullpage}
\usepackage{color}
\usepackage[reqno]{amsmath}
\usepackage{amsthm}
\usepackage{fancyvrb}
\usepackage{amssymb,enumerate}
\usepackage[all]{xy}
\usepackage{endnotes}
\usepackage{lscape}
\newtheorem{com}{Comment}
\usepackage{float}
\usepackage{hyperref}
\newtheorem{lem} {Lemma}
\newtheorem{prop}{Proposition}
\newtheorem{thm}{Theorem}
\newtheorem{defn}{Definition}
\newtheorem{cor}{Corollary}
\newtheorem{obs}{Observation}
\usepackage[compact]{titlesec}
\usepackage{dcolumn}
\usepackage{tikz}
\usetikzlibrary{arrows}
\usepackage{multirow}
\usepackage{xcolor}
\newcolumntype{.}{D{.}{.}{-1}}
\newcolumntype{d}[1]{D{.}{.}{#1}}
\definecolor{light-gray}{gray}{0.65}
\usepackage{url}
\usepackage{listings}
\usepackage{color}

\definecolor{codegreen}{rgb}{0,0.6,0}
\definecolor{codegray}{rgb}{0.5,0.5,0.5}
\definecolor{codepurple}{rgb}{0.58,0,0.82}
\definecolor{backcolour}{rgb}{0.95,0.95,0.92}

\lstdefinestyle{mystyle}{
	backgroundcolor=\color{backcolour},   
	commentstyle=\color{codegreen},
	keywordstyle=\color{magenta},
	numberstyle=\tiny\color{codegray},
	stringstyle=\color{codepurple},
	basicstyle=\footnotesize,
	breakatwhitespace=false,         
	breaklines=true,                 
	captionpos=b,                    
	keepspaces=true,                 
	numbers=left,                    
	numbersep=5pt,                  
	showspaces=false,                
	showstringspaces=false,
	showtabs=false,                  
	tabsize=2
}
\lstset{style=mystyle}
\newcommand{\Sref}[1]{Section~\ref{#1}}
\newtheorem{hyp}{Hypothesis}

\title{Problem Set 3}
\date{Due: November 19, 2022}
\author{Applied Stats/Quant Methods 1}


\begin{document}
	\maketitle
	\section*{Instructions}
	\begin{itemize}
		\item Please show your work! You may lose points by simply writing in the answer. If the problem requires you to execute commands in \texttt{R}, please include the code you used to get your answers. Please also include the \texttt{.R} file that contains your code. If you are not sure if work needs to be shown for a particular problem, please ask.
	\item Your homework should be submitted electronically on GitHub.
	\item This problem set is due before 23:59 on Sunday November 19, 2023. No late assignments will be accepted.

	\end{itemize}

		\vspace{.25cm}
	
\noindent In this problem set, you will run several regressions and create an add variable plot (see the lecture slides) in \texttt{R} using the \texttt{incumbents\_subset.csv} dataset. Include all of your code.

	\vspace{.5cm}
\section*{Question 1}
\vspace{.25cm}
\noindent We are interested in knowing how the difference in campaign spending between incumbent and challenger affects the incumbent's vote share. 
	\begin{enumerate}
		\item Run a regression where the outcome variable is \texttt{voteshare} and the explanatory variable is \texttt{difflog}.\\
		
		\vspace{.15cm}
		\lstinputlisting[language=R, firstline=38, lastline=40]{PS3_Yuanyuan_Liu.R} 
		\vspace{.15cm}
		
		\noindent The regression about voteshare and difflog:\\
		Call:
		lm(formula = voteshare ~ difflog, data = inc.sub)\\
		\vspace{.15cm}
		Residuals:\\
		Min       1Q   Median       3Q      Max\\ 
		-0.26832 -0.05345 -0.00377  0.04780  0.32749 \\
		\vspace{.15cm}
		Coefficients:\\
		Estimate Std. Error t value Pr($\left| t \right|$)  \\  
		(Intercept) 0.579031   0.002251  257.19   $<$2e-16 ***\\
		difflog     0.041666   0.000968   43.04   $<$2e-16 ***\\
		---
		Signif. codes:  0 ‘***’ 0.001 ‘**’ 0.01 ‘*’ 0.05 ‘.’ 0.1 ‘ ’ 1\\
		\vspace{.15cm}
		Residual standard error: 0.07867 on 3191 degrees of freedom\\
		Multiple R-squared:  0.3673,	Adjusted R-squared:  0.3671 \\
		F-statistic:  1853 on 1 and 3191 DF,  p-value: $<$ 2.2e-16\\
		
		\noindent The intercept is estimated to be 0.579031, and the coefficient for difflog is estimated to be 0.041666.Both coefficients are highly statistically significant (p-value < 0.001), indicating a significant relationship between the explanatory variable difflog and the outcome variable voteshare.
		The R-squared values suggest that approximately 36.73% of the variability in voteshare is explained by the linear regression model.
		The F-statistic is highly significant, indicating that the model as a whole is statistically significant.
		
		
		
		\item Make a scatterplot of the two variables and add the regression line. 		
		\vspace{.15cm}
		\lstinputlisting[language=R, firstline=42, lastline=52]{PS3_Yuanyuan_Liu.R} 
		\vspace{.15cm}
		\begin{figure}[h!]\centering
			
			\caption{\footnotesize Scatterplot of relationship between \texttt{voteshare}  and  \texttt{difflog}.}
			\label{fig:plot_1}
			\includegraphics[width=.85\textwidth]{plot_1.pdf}
		\end{figure}
		\noindent The scatterplot illustrates the relationship between campaign spending (difflog) and the incumbent's voteshare . The regression line suggests a positive correlation, indicating that as campaign spending increases, the incumbent's vote share tends to increase. The points are tightly clustered around the regression line, suggesting a strong linear relationship. However, a few outliers are noticeable, warranting further investigation. 
		\vspace{80.15cm}
		\item Save the residuals of the model in a separate object.
		
		\vspace{.15cm}
		\lstinputlisting[language=R, firstline=54, lastline=56]{PS3_Yuanyuan_Liu.R} 
		\vspace{.15cm}
		\noindent Save the residuals in a separate object:\\
		Named num [1:3193] -0.000423 -0.031684 -0.004551 0.038669 0.035529 ...\\
		- attr(*, "names")= chr [1:3193] "1" "2" "3" "4" ...\\
		Length  Class   Mode \\
		0   NULL   NULL \\
		\item Write the prediction equation.
		
		\vspace{.15cm}
		\lstinputlisting[language=R, firstline=58, lastline=66]{PS3_Yuanyuan_Liu.R} 
		\vspace{.15cm}
		\noindent Thr prediction equation:\\
		\vspace{.15cm}
		voteshare = 0.579 + 0.0417  * difflog\\
		\noindent The intercept of 0.579 indicates that the predicted vote share is 0.579 when the campaign spending difference is zero. 
		The coefficient for difflog (0.0417) predicts a unit increase in difflog is associated with 0.0417 increase in vote share.
		
	\end{enumerate}
	
\vspace{.5cm}

\section*{Question 2}
\noindent We are interested in knowing how the difference between incumbent and challenger's spending and the vote share of the presidential candidate of the incumbent's party are related.	\vspace{.25cm}
	\begin{enumerate}
		\item Run a regression where the outcome variable is \texttt{presvote} and the explanatory variable is \texttt{difflog}.	
		
		\vspace{.15cm}
		\lstinputlisting[language=R, firstline=70, lastline=72]{PS3_Yuanyuan_Liu.R} 
		\vspace{.15cm}
		\noindent Call:
		lm(formula = presvote ~ difflog, data = inc.sub)\\
		\vspace{.15cm}
		Residuals:\\
		Min       1Q   Median       3Q      Max \\
		-0.32196 -0.07407 -0.00102  0.07151  0.42743\\ 
		\vspace{.15cm}
		Coefficients:
		Estimate Std. Error t value Pr($>|t|$)\\    
		(Intercept) 0.507583   0.003161  160.60   $<$2e-16 ***\\
		difflog     0.023837   0.001359   17.54   $<$2e-16 ***\\
		---\\
		Signif. codes:  0 ‘***’ 0.001 ‘**’ 0.01 ‘*’ 0.05 ‘.’ 0.1 ‘ ’ 1\\
		\vspace{.15cm}
		Residual standard error: 0.1104 on 3191 degrees of freedom\\
		Multiple R-squared:  0.08795,	Adjusted R-squared:  0.08767 \\
		F-statistic: 307.7 on 1 and 3191 DF,  p-value: $<$ 2.2e-16\\
		\vspace{.15cm}
		\item Make a scatterplot of the two variables and add the regression line. 	
		\vspace{.15cm}
		\lstinputlisting[language=R, firstline=74, lastline=83]{PS3_Yuanyuan_Liu.R} 
		\vspace{.15cm}
		\begin{figure}[h!]\centering
			
			\caption{\footnotesize Scatterplot of relationship between \texttt{presvote}and \texttt{difflog}.}
			\label{fig:plot_2}
			\includegraphics[width=.85\textwidth]{plot_2.pdf}
		\end{figure}
		\noindent This scatterplot provides a visual representation of the relationship between campaign spending difference (difflog) and presidential vote (presvote). The regression line helps to identify the general trend in the data. 
		\item Save the residuals of the model in a separate object.	
		\vspace{.15cm}
		\lstinputlisting[language=R, firstline=85, lastline=87]{PS3_Yuanyuan_Liu.R} 
		\vspace{.15cm}
		\noindent Save the residuals in a separate object residuals\textunderscore2:\\
		Named num [1:3193] 0.00561 0.03758 -0.05313 -0.05299 -0.04584 ...\\
		- attr(*, "names")= chr [1:3193] "1" "2" "3" "4" ...\\
		NULL
		\item Write the prediction equation.
		
		\vspace{.15cm}
		\lstinputlisting[language=R, firstline=89, lastline=92]{PS3_Yuanyuan_Liu.R} 
		\vspace{.15cm}
		\noindent The prediction equation between presvote and difflog:\\
		presvote= 0.5076 + 0.0238 *difflog
		\noindent The intercept of 0.5076 indicates that the presidential vote is 0.5076 when the campaign spending difference is zero. 
		The coefficient for difflog (0.0238) predicts a unit increase in difflog is associated with 0.0238 increase in presidential vote.
	\end{enumerate}
	
\vspace{.5cm}
\section*{Question 3}

\noindent We are interested in knowing how the vote share of the presidential candidate of the incumbent's party is associated with the incumbent's electoral success.
	\vspace{.25cm}
	\begin{enumerate}
		\item Run a regression where the outcome variable is \texttt{voteshare} and the explanatory variable is \texttt{presvote}.
			
			\vspace{.15cm}
			\lstinputlisting[language=R, firstline=96, lastline=98]{PS3_Yuanyuan_Liu.R} 
			\vspace{.15cm}
			\noindent The regression about voteshare and prevote:\\
			Call:
			lm(formula = voteshare ~ presvote, data = inc.sub)\\
			
			Residuals:\\
			Min       1Q   Median       3Q      Max \\
			-0.27330 -0.05888  0.00394  0.06148  0.41365 \\
			
			Coefficients:\\
			Estimate Std. Error t value Pr($>|t|$)    \\
			(Intercept) 0.441330   0.007599   58.08   $<$2e-16 ***\\
			presvote    0.388018   0.013493   28.76   $<$2e-16 ***\\
			---\\ 
			Signif. codes:  0 ‘***’ 0.001 ‘**’ 0.01 ‘*’ 0.05 ‘.’ 0.1 ‘ ’ 1\\
			
			Residual standard error: 0.08815 on 3191 degrees of freedom\\
			Multiple R-squared:  0.2058,	Adjusted R-squared:  0.2056 \\
			F-statistic:   827 on 1 and 3191 DF,  p-value: $<$ 2.2e-16\\
		\item Make a scatterplot of the two variables and add the regression line. 
		
		\vspace{.15cm}
		\lstinputlisting[language=R, firstline=100, lastline=109]{PS3_Yuanyuan_Liu.R} 
		\vspace{.15cm}
		\begin{figure}[h!]\centering
			
			\caption{\footnotesize Scatterplot of relationship between \texttt{voteshare} and \texttt{presvote}.}
			\label{fig:plot_3}
			\includegraphics[width=.85\textwidth]{plot_3.pdf}
		\end{figure}
		
		\item Write the prediction equation.
		
		\vspace{.15cm}
		\lstinputlisting[language=R, firstline=111, lastline=113]{PS3_Yuanyuan_Liu.R} 
		\vspace{.15cm}
		
	\end{enumerate}
	

\vspace{.5cm}
\section*{Question 4}
\noindent The residuals from part (a) tell us how much of the variation in \texttt{voteshare} is $not$ explained by the difference in spending between incumbent and challenger. The residuals in part (b) tell us how much of the variation in \texttt{presvote} is $not$ explained by the difference in spending between incumbent and challenger in the district.
	\begin{enumerate}
		\item Run a regression where the outcome variable is the residuals from Question 1 and the explanatory variable is the residuals from Question 2.	
		
		\vspace{.15cm}
		\lstinputlisting[language=R, firstline=117, lastline=119]{PS3_Yuanyuan_Liu.R} 
		\vspace{.15cm}
		
		\item Make a scatterplot of the two residuals and add the regression line. 	
		\vspace{.15cm}
		\lstinputlisting[language=R, firstline=121, lastline=128]{PS3_Yuanyuan_Liu.R} 
		\vspace{.15cm}
		\begin{figure}[h!]\centering
			
			\caption{\footnotesize Scatterplot of relationship between \texttt{residuals\textunderscore1} and \texttt{residuals\textunderscore2}.}
			\label{fig:plot_4}
			\includegraphics[width=.85\textwidth]{plot_4.pdf}
		\end{figure}
		\item Write the prediction equation.
		
		\vspace{.15cm}
		\lstinputlisting[language=R, firstline=131, lastline=134]{PS3_Yuanyuan_Liu.R} 
		\vspace{.15cm}
		
	\end{enumerate}
	
\vspace{.5cm}

\section*{Question 5}
\noindent What if the incumbent's vote share is affected by both the president's popularity and the difference in spending between incumbent and challenger? 
	\begin{enumerate}
		\item Run a regression where the outcome variable is the incumbent's \texttt{voteshare} and the explanatory variables are \texttt{difflog} and \texttt{presvote}.
		
		\vspace{.15cm}
		\lstinputlisting[language=R, firstline=138, lastline=140]{PS3_Yuanyuan_Liu.R} 
		\vspace{.15cm}
		
		\item Write the prediction equation.	
		\vspace{.15cm}
		\lstinputlisting[language=R, firstline=142, lastline=147]{PS3_Yuanyuan_Liu.R} 
		\vspace{.15cm}
		
		\item What is it in this output that is identical to the output in Question 4? Why do you think this is the case?
		
		\vspace{.15cm}
		\noindent
		\vspace{.15cm}
		
	\end{enumerate}




\end{document}
